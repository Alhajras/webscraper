
%------------------------------------------------------------------------------
%       package includes
%------------------------------------------------------------------------------
    % font encoding is set up for pdflatex, for other environments see
    % http://tex.stackexchange.com/questions/44694/fontenc-vs-inputenc
    \usepackage[T1]{fontenc}  % 8-bit fonts, improves handling of hyphenations
    \usepackage[utf8x]{inputenc}
    % provides `old' commands for table of contents. Eases the ability to switch
    % between book and scrbook
    \usepackage{scrhack}


    % ------------------- layout, default -------------------
    % adjust the style of float's captions, separated from text to improve readabilty
    \usepackage[labelfont=bf, labelsep=colon, format=hang, textfont=singlespacing]{caption}
    % With format = hang your caption will look like this:
    % Figure 1: Lorem ipsum dolor sit amet,
    %           consectetuer adipiscing elit.
    %           Ut purus elit, vestibulum
    % If you instead want
    % Figure 1: Lorem ipsum dolor sit amet,
    % consectetuer adipiscing elit. Ut purus
    % elit, vestibulum
    % change to format=plain
    \usepackage{chngcntr}  % continuous numbering of figures/tables over chapters
    \counterwithout{equation}{chapter}
    \counterwithout{figure}{chapter}
    \counterwithout{table}{chapter}

    % Uncomment the following line if you switch from scrbook to book
    % and comment the setkomafont line
    %\usepackage{titlesec}  % remove "Chapter" from the chapter title
    %\titleformat{\chapter}[hang]{\bfseries\huge}{\thechapter}{2pc}{\huge}
    \setkomafont{chapter}{\normalfont\bfseries\huge}

    \usepackage{setspace}  % Line spacing
    \onehalfspacing
    % \doublespacing  % uncomment for double spacing, e.g. for annotations in correction

    % ------------------- functional, default-------------------
    \usepackage[dvipsnames]{xcolor}  % more colors
    \usepackage{array}  % custom format per column in table - needed on the title page
    \usepackage{graphicx}  % include graphics
    \usepackage{subfig}  % divide figure, e.g. 1(a), 1(b)...
    \usepackage{amsmath}  % |
    \usepackage{amsthm}   % | math, bmatrix etc
    \usepackage{amsfonts} % |
    \usepackage{calc}  % calculate within LaTeX
    \usepackage[unicode=true,bookmarks=true,bookmarksnumbered=true,
                bookmarksopen=true,bookmarksopenlevel=1,breaklinks=false,
                pdfborder={0 0 0},backref=false,colorlinks=false]{hyperref}
    \usepackage{etoolbox} % if-else commands
\documentclass{scrreprt}
\usepackage{color}
\definecolor{editorGray}{rgb}{0.95, 0.95, 0.95}
\definecolor{editorOcher}{rgb}{1, 0.5, 0} % #FF7F00 -> rgb(239, 169, 0)
\definecolor{editorGreen}{rgb}{0, 0.5, 0} % #007C00 -> rgb(0, 124, 0)
\usepackage{upquote}
\usepackage{listings}
\lstdefinelanguage{JavaScript}{
  morekeywords={typeof, new, true, false, catch, function, return, null, catch, switch, var, if, in, while, do, else, case, break},
  morecomment=[s]{/*}{*/},
  morecomment=[l]//,
  morestring=[b]",
  morestring=[b]'
}

\lstdefinelanguage{HTML5}{
        language=html,
        sensitive=true, 
        alsoletter={<>=-},
        otherkeywords={
        % HTML tags
        <html>, <head>, <title>, </title>, <meta, />, </head>, <body>,
        <canvas, \/canvas>, <script>, </script>, </body>, </html>, <!, html>, <style>, </style>, ><
        },  
        ndkeywords={
        % General
        =,
        % HTML attributes
        charset=, id=, width=, height=,
        % CSS properties
        border:, transform:, -moz-transform:, transition-duration:, transition-property:, transition-timing-function:
        },  
        morecomment=[s]{<!--}{-->},
        tag=[s]
}

\lstset{%
    % Basic design
    backgroundcolor=\color{editorGray},
    basicstyle={\small\ttfamily},   
    frame=l,
    % Line numbers
    xleftmargin={0.75cm},
    numbers=left,
    stepnumber=1,
    firstnumber=1,
    numberfirstline=true,
    % Code design   
    keywordstyle=\color{blue}\bfseries,
    commentstyle=\color{darkgray}\ttfamily,
    ndkeywordstyle=\color{editorGreen}\bfseries,
    stringstyle=\color{editorOcher},
    % Code
    language=HTML5,
    alsolanguage=JavaScript,
    alsodigit={.:;},
    tabsize=2,
    showtabs=false,
    showspaces=false,
    showstringspaces=false,
    extendedchars=true,
    breaklines=true,        
    % Support for German umlauts
    literate=%
    {Ö}{{\"O}}1
    {Ä}{{\"A}}1
    {Ü}{{\"U}}1
    {ß}{{\ss}}1
    {ü}{{\"u}}1
    {ä}{{\"a}}1
    {ö}{{\"o}}1
}

    \usepackage{rotating}
    \usepackage{tikz}
    %==========================================
    % You might not need the following packages, I only included them as they
    % are needed for the example floats
    % ------------------- functional, custom -------------------
   %  \usepackage{algorithm,algpseudocode} % THIS OHNE
    \usepackage{algpseudocode}
    \usepackage{algorithm}
    \usepackage{bm}  % bold greek variables (boldmath)
    \usepackage{tikz}
    \usetikzlibrary{positioning}  % use: above left of, etc
    
    % required for the ToDo list
    \usepackage{ifthen}

    % Improves general appearance of the text
    %\usepackage[protrusion=true,expansion=true, kerning]{microtype}
    %\usepackage{enumitem}
    % nicer font for pdf rendering
    %\usepackage{lmodern}
    
    % For nicer looking tables
    \usepackage{booktabs}

    % usually you don't need this, just for demonstration of a longer caption
    \usepackage{lipsum}

%------------------------------------------------------------------------------
%       (re)new commands / settings
%------------------------------------------------------------------------------
    % ----------------- referencing ----------------
    \newcommand{\secref}[1]{Section~\ref{#1}}
    \newcommand{\chapref}[1]{Chapter~\ref{#1}}
    \renewcommand{\eqref}[1]{Equation~(\ref{#1})}
    \newcommand{\figref}[1]{Figure~\ref{#1}}
    \newcommand{\tabref}[1]{Table~\ref{#1}}

    % ------------------- colors -------------------
    \definecolor{darkgreen}{rgb}{0.0, 0.5, 0.0}
    % Colors of the Albert Ludwigs University as in
    % https://www.zuv.uni-freiburg.de/service/cd/cd-manual/farbwelt
    \definecolor{UniBlue}{RGB}{0, 74, 153}
    \definecolor{UniRed}{RGB}{193, 0, 42}
    \definecolor{UniGrey}{RGB}{154, 155, 156}


    % ------------------- layout -------------------
    % prevents floating objects from being placed ahead of their section
    \let\mySection\section\renewcommand{\section}{\suppressfloats[t]\mySection}
    \let\mySubSection\subsection\renewcommand{\subsection}{\suppressfloats[t]\mySubSection}



    % ------------------- math formatting commands -------------------
    % define vectors to be bold instead of using an arrow
    \renewcommand{\vec}[1]{\mathbf{#1}}
    \newcommand{\mat}[1]{\mathbf{#1}}
    % tag equation with name
    \newcommand{\eqname}[1]{\tag*{#1}}


    % ------------------- pdf settings -------------------
    % ADAPT THIS
    \hypersetup{pdftitle={\thetitle},
                pdfauthor={\theauthor},
                pdfsubject={Undergraduate thesis at the Albert Ludwig University of Freiburg},
                pdfkeywords={deep learning, awesome algorithm,  undergraduate thesis},
                pdfpagelayout=OneColumn, pdfnewwindow=true, pdfstartview=XYZ, plainpages=false}


    %==========================================
    % You might not need the following commands, I only included them as they
    % are needed for the example floats

    % ------------------- Tikz styles -------------------
    \tikzset{>=latex}  % arrow style


    % ------------------- algorithm ---------------------
    % Command to align comments in algorithm
    \newcommand{\alignedComment}[1]{\Comment{\parbox[t]{.35\linewidth}{#1}}}
    % define a foreach command in algorithms
    \algnewcommand\algorithmicforeach{\textbf{foreach}}
    \algdef{S}[FOR]{ForEach}[1]{\algorithmicforeach\ #1\ \algorithmicdo}

    % line spacing should be 1.5
    \renewcommand{\baselinestretch}{1.3}

    % set distance between items in a list, for more details see the
    % enumitem package: https://www.ctan.org/pkg/enumitem
%    \setlist{itemsep=.5em}
    
    % use ra in your tables to increase the space between rows
    % 1.3 should be fine
    \newcommand{\ra}[1]{\renewcommand{\arraystretch}{#1}}

	% ToDo counters
	\usepackage{ifthen} %für whiledo-Schleife
	\newcounter{todos}
	\setcounter{todos}{0}
	\newcounter{extends}
	\setcounter{extends}{0}
	\newcounter{drafts}
	\setcounter{drafts}{0}

	% ------------------- marker commands -------------------
    % ToDo command
    \newcommand{\todo}[1]{\textbf{\textcolor{red}{(TODO: #1)}}\refstepcounter{todos}\label{todo \thetodos}}
	\newcommand{\extend}[1]{\textbf{\textcolor{darkgreen}{(EXTEND: #1)}}\refstepcounter{extends}\label{extend \theextends}}
	% Lighter color to note down quick drafts
	\newcommand{\draft}[1]{\textbf{\textcolor{NavyBlue}{(DRAFT: #1)}}\refstepcounter{drafts}\label{draft \thedrafts}}
	%% Unit systems.
    \usepackage[binary-units=true]{siunitx}
    \usepackage{romannum}
	\DeclareSIUnit{\million}{\text{M}}
    \DeclareSIUnit{\billion}{\text{B}}
	% microtype with lmodern, see https://tex.stackexchange.com/questions/75305/microtype-warning-with-lmodern-package-and-koma-script
	%\DeclareMicrotypeAlias{lmss}{cmr}