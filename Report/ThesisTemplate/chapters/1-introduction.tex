\chapter{Introduction}
\label{chap:introduction}
\section{Motivation}


Since the beginning of the Digital Revolution, known as the Third Industrial Revolution, in the latter half of the 20th century, the importance of data has increased as it became the new currency shaping the dynamics of our interconnected world. From social media platforms and e-commerce transactions to information sharing and entertainment consumption, online activities generate enormous amounts of data. The online data is sometimes referred to as the 'new oil' or the 'new currency', as it impacts almost the same economies and societies as oil. Businesses and organizations understand the power of data as they provide insight into consumer behaviour, refine business strategies, and enhance decision-making processes. Furthermore, the rise of artificial intelligence has further amplified the value of Internet data. Natural language processing (NLP)\footnote{Natural language processing is an interdisciplinary field that enables computers to understand and manipulate speech.} is becoming a new hot topic as all the giant firms race to create their model; however, data is the fuel to power those models. The more data is collected, the better the model can become. Consequently, collecting, analyzing, and leveraging internet data has become a cornerstone of competitiveness, innovation, and progress in the digital age.

Internet data can be harvested by using automated software programs called Web crawlers, also known as web spiders or web bots. Their main goal is to discover, retrieve, and index\footnote{Indexing involves the storage of document indexes to enhance the speed and performance of locating relevant documents in response to a search query.} information from websites. The applications and use cases of internet crawlers are diverse and valuable, to name a few: 

\begin{itemize}
  \item \textbf{Search Engines}: Crawlers are essential components to build any search engine, such as Google, Bing, and Yahoo. Crawlers are run on supercomputers to crawl all the content on the internet index web pages and gather information about content, keywords, and links. This data is then used to rank and display search results, ensuring users can quickly find relevant information.
  \item \textbf{Market Research}: Businesses use web crawlers to collect data about their competitors, market trends, and consumer opinion. This information helps in making informed business decisions.
  \item \textbf{Fraud Detection}: Cybersecurity companies use crawlers to catch fraudulent activities by monitoring online transactions, identifying unusual patterns, and tracking potential threats.
  \item \textbf{Content Monitoring}: E-commerce platforms utilize crawlers to extract product prices from various websites. This enables them to offer consumers real-time price comparisons and assist in finding the best deals. Moreover, social media platforms use crawlers to monitor their content to prevent unwanted posts and images.
\end{itemize}

\section{Problem Statement}

The World Wide Web (WWW) contains enormous data that escalates with each passing day. The total data created and replicated is expected to grow to more than 180 zettabytes by 2025 according to Statista. This upward trajectory is expected to continue due to the growing affordability of smartphones and the broadening reach of internet accessibility. Moreover, due to the COVID-19 pandemic, more companies started offering remote work, more local shops transformed into online stores, and more services switched to cloud-based. This social evolution over recent years has embedded the Internet as an integral cornerstone of our daily life.

The expansion of the Internet gives rise to an immense overflow of data, resulting in a noise that complicates the task of locating relevant information for both end users and organizational queues. To surmount this hurdle, the concept of Information Retrieval (IR) was coined by Calvin Mooers in 1951. IR involves the art of accessing and recovering data from an extensive pool of unstructured information. A particularly pragmatic manifestation of IR involves the extraction of data from the Internet, thus advocating the implementation of a universal algorithm for procuring and categorizing requisite information. In this pursuit, crawlers or spiders emerge as automated entities designed to adhere to predefined directives, allowing the automated fetching and extracting of data from the Internet.

One form of IR is a web search engine. A web search engine is a system engineered to index the Internet. Users can search for articles, documents and pages by entering keywords. The search will provide a list of the most related result that matches the search query. Using the crawlers explained earlier; the engine can index the collected information and optimize the search process using different algorithms and techniques. 
 
Although search engines such as Google, Bing and DuckDuckGo display remarkable proficiency in their web crawling and indexing capabilities, specific businesses, like those in E-commerce, have a distinct interest in demonstrating the most competitive pricing for a given product, a key insight into their competitive landscape. However, more than a straightforward Google search is needed, as the search query index and rank the documents on the Internet based on Google's vendor parameters. These parameters include preeminent brand visibility, user geolocation, SEO optimization proficiency, and hidden variables, excluding the lowest price criterion from the page ranking equation. A second issue arises from the format of search results, with each search engine providing a distinctive result format. Companies may want to exclude some portion of the Internet from the index and rank. Also, they should prioritize some pages more than others.

The previous requirements are tiny use case, among others, that limits companies from using a simple search on Google. Search engines need to be tweaked and configured to match the domain of interest as E-commerce in the previous example and to match a specific use case like the price comparison mentioned.

The main concern is that businesses are often interested in only a portion of the Internet that intersects with their domain and expertise. Furthermore, the criteria for indexing and page ranking depend heavily on their use case and is vital to their business to take control of it and configure it as fit. Hiring domain expertise is inevitable for any business. However, the data scientists often have to go throw some basics steps to get their crawler up and running; those steps cost money and time; it would be helpful to have an infrastructure that allows the data scientist to have starting script that can be extended easily and needs little to no programming knowledge. 




Amount of data created, consumed, and stored 2010-2020, with forecasts to 2025
Published by 
Petroc Taylor
, Sep 8, 2022
https://www.statista.com/statistics/871513/worldwide-data-created/#:~:text=The%20total%20amount%20of%20data,to%20more%20than%20180%20zettabytes.
\section{Contribution}

In this thesis, we aim to answer the following questions:

\begin{itemize}
  \item What are the challenges and bottlenecks to creating a scalable, configurable generic search engine?
\item Can we implement a basic tool that can be easily scaled to crawl different websites independently from their DOM structure?
    \item Can we create a proper User Interface UI that allows users to crawl and index targeted websites from the internet?
    \item Can we integrate the indexing and crawling processes in the same tool?
    \item Can we find meaningful evaluation metrics for the implemented search engine?
\end{itemize}
\section{Chapter Overview}
