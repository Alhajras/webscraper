\chapter{Introduction}
\label{chap:introduction}
\section{Motivation}

The World Wide Web (WWW) contains an enormous amount of data; this data is increasing each day rapidly. The amount of total data created and replicated is expected to grow to more than 180 zettabytes by 2025 according to Statista. The growth is expected to continue as more smartphones are more and more affordable, and more people can reach the internet. Moreover, due to the COVID-19 pandemic, more companies started offering work remotely, more shops created online stores, and more services switched to cloud-based. This change in society during the last few years has made the internet a vital part of our day-to-day life.   

Although the data is available, making a helpful meaning is a challenge. Search engines, for example, try to organize and index that information to make them easily searchable by the end user. Furthermore, collecting data can help spot competitors and have a deeper meaning in the market. Additionally, data scientists are now playing essential roles in most organizations and enterprises to understand consumer needs by collecting and analyzing data from the web.  


Although some websites provide APIs to provide organized information about their services, for example, some airline companies provide API that serves information about their flight schedules, other online shops also provide a documented API to get helpful information about their available products. This is not a guaranteed approach to gathering data, as not all websites offer an excellent documented API. For example, social media websites are reluctant to give information about their users, which is understandable. What if you would like to go through all comments and classify them as spam or not? Depending only on the assumption of having an API for each website is a fragile approach. 

Information retrieval (IR) is a term introduced in 1951 by Calvin
Mooers. It is accessing and retrieving data from a vast pool of unstructured information. The generic use case of the IR is to 

Amount of data created, consumed, and stored 2010-2020, with forecasts to 2025
Published by 
Petroc Taylor
, Sep 8, 2022
https://www.statista.com/statistics/871513/worldwide-data-created/#:~:text=The%20total%20amount%20of%20data,to%20more%20than%20180%20zettabytes.
\section{Contributions}
\section{Chapter Overview}
